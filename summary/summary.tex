\chapter*{Summary}

This monograph represents a series of spectroscopic studies aimed at a comprehensive description of the storage and transport of elementary charges in redox conductive polymers, that have applications in electrochemical energy storage devices. A specific class of TEMPO-Salen polymers is considered. In the beginning, we present an overview of the available charge-transport and charge-storage models for redox conductive polymers and indicate, how the models can be refined by using the toolbox of spin resonance spectroscopy. We then describe the spectroscopic and electrochemical methods that will be used to obtain the information on the undisclosed charge transport and storage mechanisms. Next chapter is devoted to the fabrication of a TEMPO-Salen electrochemical cell inside an X-Band EPR sample tube, that is used for operando spectroscopic experiments. The discussion of the operando spectroscopic data takes place in the next chapter. The chapter after that describes a magnetic resonance experiment with electrical detection on the working electrochemical cell. Then, we focus on the application of pulsed EPR techniques to study domain formation in the redox conductive polymer films. We will further consider the attempts to observe electrically detected magnetic resonance signals in a slowly charging TEMPO-Salen electrochemical cell. Afterwards, we present and discuss the deep-trap-dominated semiconductor model of storage and transport of charge in densely packed redox conductive polymers. Finally, in the Chapter Conclusions and Outlook, we summarize the monograph and sketch a roadmap for the future investigations.
