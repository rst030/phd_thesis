\chapter*{Appendix A:\\Electron Magnetic Resonance Environment}
A python-based interactive program EMRE (Electron Magnetic Resonance Environment) was developed to interconnect and control a cwEPR spectrometer and to simultaneously operate the electrochemical device under testing. The program allows one to control a potentiostat and perform electrochemical measurements such as cyclic voltammetry, potentiostatic and galvanostatic charge-discharge cycling, while recording cwEPR spectra with customizable, lab-built and commercial EPR spectrometers. The multi-threading architecture of EMRE allows for simultaneous visualization of data. Most of the redox conductive films considered in this thesis were manufactured using EMRE (Fig.~\ref{fig:Figure_1}). EMRE was used to control the SoC of the pDiTBUS films for the cryogenic pulsed EPR measurements (Fig.~\ref{fig:Figure_1},Fig.~\ref{fig:Figure_S27}). Some of the potential-dependent (Fig.~\ref{fig:cwEPR_RT_NiSalen_OPERANDO}) and degradation spectroelectrochemical series (Fig.~\ref{fig:operando_degradation_3_lines_release}) were carried out using EMRE installed on a lab-built X-band cwEPR spectrometer ``Lyra''.  EMRE was installed on a commercial table-top cwEPR spectrometer ``Magnettech'' for recording a series of cryogenic cwEPR measurements interlaced with charge-discharge cycling at room temperature.\\ 

\par
EMRE is free to redistribute and modify. The source code of the program and the documentation are available at the remote repository:\\ 

\par
\texttt{https://github.com/rst030/EMRE}\\
 
\par
The multiprocessing capability and the visualizing toolbox of EMRE was further developed in the framework of the open-source magnetic resonance imaging initiative at the Physikalisch-Technische Bundesanstalt:\\

\par
\texttt{https://github.com/rst030/cosi-measure/Software}