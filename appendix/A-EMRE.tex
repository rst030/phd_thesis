\chapter*{Appendix A:\\Electron Magnetic Resonance Environment}
A python-based interactive program EMRE (Electron Magnetic Resonance Environment) was developed to interconnect and control the EPR spectrometer and to simultaneously operate the electrochemical device under testing. The program allows one to perform electrochemical measurements such as cyclic voltammetry and galvanostatic charge-discharge cycling and to simultaneously record cwEPR spectra with customizable, lab-built and commercial EPR spectrometers. Most of the redox conductive films considered in this thesis were manufactured using EMRE. Additionally, the some of the potential-dependent spectra and degradation studies were carried out using EMRE. The source code of the program and the documentation is available at the remote repository:\\ \texttt{https://github.com/rst030/EMRE}.