\chapter*{Appendix D:\\Experimental Details}
\label{Experimental_Section}

\section*{Electrochemistry}
\par
CV curves were measured either in a 5~ml beaker using the on-substrate WE, a coiled Pt CE and an Ag/AgNO\textsubscript{3} RE with the above mentioned electrolyte (Fig.~\ref{fig:Figure_1}c) or in a modified EPR tube for in-operando measurements (Fig.~\ref{fig:Figure_1}d), with the on-substrate WE and CE and a Ag or Ag/AgCl wire as a RE with the same electrolyte. In-operando SEC cwEPR measurements were done by holding the sample at the redox potential of interest for 200~s using the chronoamperometry mode on the Keithley 2450-EC potentiostat, while the cwEPR scans were accumulated. The 200~s scans allowed for cwEPR spectra with a suitable S/N without causing any noticeable degradation to the electrochemical cell up to a potential of 950~mV.



\section*{cwEPR Measurements}

The cwEPR spectra in Figures~\ref{fig:operando_cold_cycle},\ref{fig:freezing_of_pditbus_battery_1D},\ref{fig:operando_cold_battery},\ref{
fig:Figure_7},\ref{fig:operando_solid_battery},\ref{fig:repeated_cycling_degradation} were recorded at X-band frequencies ($\sim$9.4~GHz) using a laboratory-built EPR spectrometer. For some spectra the magnetic field axes were rescaled to a microwave frequency of 9.6~GHz in order to facilitate comparability between different spectra. The magnetic field was regulated using a Bruker BH15 field controller and monitored with a Bruker ER 035M NMR Gaussmeter while a Bruker ER 041 MR microwave bridge (with a ER 048 R microwave controller) was used for microwave generation (200~mW source) and detection (diode-detection). The static magnetic field was modulated at 100~kHz and lock-in detection was carried out using a Stanford Research SR810 lock-in amplifier in combination with a Wangine WPA-120 audio amplifier. The lock-in detection leads to the derivative spectra. An ESR 910 helium flow cryostat together with an ITC503 temperature controller (Oxford Instruments, UK) was used for low-temperature measurements. $g$-factor calibration was additionally done using a N@C\textsubscript{60} powder sample at room temperature, with $g= $2.00204.\cite{si_Wittmann2018_JMagnReson} SEC cwEPR spectra were taken using the setup described earlier, using a on-substrate electrode setup, with a polymer film deposited on the WE using electropolymerization, and modified EPR tube where the electrolyte was 0.1~\textsc{m} solution of Et\textsubscript{4}NBF\textsubscript{4} in ACN ($\epsilon\approx40$, T\textsubscript{melt}$=$227~K). Control over the redox potential of the cell was achieved by using a Keithley 2450-EC potentiostat.\\

\par
The potential-dependent cwEPR spectra were recorded at room temperature for in-operando measurements and at 150~K for ex-situ measurements without electrolyte. The low temperature measurements provide an increased S/N for the cwEPR spectra and therefore allowed for a better deconvolution of the spectral components contributing to the signal. The potential-dependent in-operando cwEPR spectra were measured for p-DiTS, while the ex-situ cwEPR spectra were measured for both p-DiTS and p-NiMeOSalen films.

\section*{Pulsed EPR Measurements}
\par
All pulsed EPR measurements were carried out at X-band (9.6~GHz) using a Bruker Elexsys E~580 spectrometer equipped with a 1~kW TWT microwave amplifier (Applied System Engineering, USA) and an ER~4118 X-MD5 dielectric ring resonator. Temperature control was achieved with the use of a helium flow cryostat (CF935) and an ITC5 temperature controller (Oxford instruments, UK). CWEPR spectra were recorded with 5~G, 100~kHz field modulation using a cw 200\,mW microwave source. The echo-detected field sweep was recorded using the Hahn echo sequence of $\frac{\pi}{2} - \tau - \pi - \tau - \textrm{echo}$ with $\frac{\pi}{2} =$~20~ns and $\tau =$~120~ns. The intensity of the refocused Hahn echo was integrated within a 60~ns integrator gate centered at the maximum of the echo time trace which arrived at $t=2\tau$ after the $\frac{\pi}{2}$ pulse. The pulse sequence was repeated each 1.02~ms to avoid possible saturation effects. The 1~kW amplifier output was attenuated by 11~dB to achieve the maximal spin echo signal.\\

The transverse relaxation times $T_m$ in Section~\ref{S:RELAX_TIMES} were measured in the echo decay experiment with a 2-pulse sequence. A sequence of $\frac{\pi}{2} - \tau - \pi - \tau - \textrm{echo}$ was applied with $\frac{\pi}{2} =$20~ns and a variable delay of 120~$<\tau<$~4000~ns. The intensity of the decaying echo was recorded as a function of the delay $\tau$ with an increment of 16~ns and plotted as a function of 2$\tau$. The integrator gate width was 60~ns. The pulse sequence with incrementing $\tau$ was repeated each 1.02~ms to avoid possible saturation effects. The echo decay transients for pDiTBuS and TEMPOL contain oscillations with a period of $\Delta (2\tau) \approx 250$~ns that are caused by the electron spin echo envelope modulation (ESEEM) at $\Delta\omega\approx 15$~MHz. This ESEEM frequency is close to the Larmor frequency of a proton. The ESEEM effect in pDiTBuS is weaker as compared to TEMPOL due to the fast decay of the signal.\\

The longitudinal relaxation times $T_1$ were measured using inversion recovery with echo detection. A sequence of $p_I - \tau - \frac{\pi}{2} - \tau - \pi - \textrm{echo}$ was applied with $p_I=$32~ns, $\frac{\pi}{2} =$20~ns and a variable 120~ns~$<\tau<$4~ms. The intensity of the decaying echo was recorded as a function of the delay $\tau$ and plotted as a function of $\tau$.

