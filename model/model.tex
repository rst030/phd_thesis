\chapter{The Deep-Trap Model of a TEMPO-Salen Electrode Film}

\paragraph*{}
A TEMPO-Salen redox conductive film can be seen as a p-type molecular semiconductor enriched with hole traps. The conductive Salen backbone is carrying positive polarons and bipolarons that delocalize within the polymer fragments and effectively hop between them. When a polaron, traveling through the Salen backbone, approaches a charge-bearing TEMPO$^{\bullet}$ fragment, it rather hops to it, recombining with the unpaired electron of the radical. The TEMPO$^{\bullet}$ oxidizes and becomes TEMPO${^+}$ which now bears a positive charge. Therefore, TEMPO$^{\bullet}$ is a trap for the positive charge carrier (hole) that is injected into the poly-Salen network.\\
The charging of a TEMPO-Salen cathode film can be seen as a consequent filling of traps in a hole-transporting semiconductor. The relative orientations of the spins of the recombining particles defines the probability of the recombination process. At low SoC, the hole, traveling through the polymer, has many TEMPO$^{\bullet}$ candidates to recombine with. Some of the TEMPO$^{\bullet}$ radicals are in the $\vert{\uparrow\rangle}$ state, some are in the $\vert{\downarrow\rangle}$ state, - so the recombination process does not depend on the spin state of the hole, as it can recombine to the radical in the ``appropriate'' spin state. At higher SoC, the more TEMPO$^{\bullet}$ become occupied with holes and become TEMPO${^+}$, the longer distance the hole needs to overcome to meet the TEMPO$^{\bullet}$ that has the ``appropriate'' spin state. In this case, the recombination process must become spin dependent and an EDMR signal appears.\\
\ik{no edmr data to prove that, yet.}
The short distance between the TEMPO$^{\bullet}$ in the TEMPO-Salen film leads to the strong exchange interaction between the neighboring TEMPO$^{\bullet}$. That leads to the anti-parallel alignment of their spins and makes them EPR silent and causes a drastic difference between the Coulomb counting and the ESOC data. The formation of bipolarons in the Salen backbone and the close packing of TEMPO$^{\bullet}$ may lead to the two-hole-two-electron recombination that can explain the high charging rates of TEMPO-Salen films.



