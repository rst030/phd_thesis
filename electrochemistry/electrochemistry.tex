\chapter{Electrochemical Energy Storage in Redox Conductive Polymers}

\paragraph*{}
Batteries based on conjugated polymers containing stable radical moieties as high-capacitance groups represent a promising class of future electrochemical power sources.\cite{nakahara2002_cpl, nishide2004_electact,xie2021_mathoriz,Rohland2022} They combine the advantages of high-power supercapacitors, namely high discharge rates, and the high energy density of conventional lithium-ion technology. In contrast to the lithium-ion battery, the charging of an organic battery does not involve intercalation of metal ions into the electrodes. This reduces the structural change of the electrode upon repeated recharging which allows for a longer cycle life. Additionally, the non-metallic electrodes of an organic battery do not experience Joule heating, and this allows for much higher charge and discharge rates. A further beneficial property of organic materials over traditional inorganic materials is their availability and the low cost of the starting materials for the synthesis of the target polymers in conjunction with good mechanical properties.\cite{janoschka2012_advmater, muench2016_chemrev, friebe2017_topcurrchem}
\par
While active electrode materials with nitroxide radicals as redox-active groups are ideally suited for organic radical batteries (ORBs) that exhibit high power densities, the broad application of most nitroxide-based materials is limited by their moderate electrical properties. A promising route towards overcoming the conductivity problem is the use of polymers that combine radical-containing moieties and a conductive backbone. This strategy was successfully followed in a number of studies focusing on different polymers.\cite{oyaizu2015_polymerjournal, bahaceci2013_jpowersources, katsumata2006_mrc, xu2014_electact, aydin2015_jsoistatelect, schwartz2018_synthmet}




