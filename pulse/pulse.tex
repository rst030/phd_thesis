\chapter{pEPR Spectroscopy of Densely Packed Nitroxide Radicals}
\label{ch:pulsed_epr}

\section{Coherent Spin Motion under Pulsed Microwave Field}
When a spin system is excited with a microwave pulse, its evolution is described with the set of equations that is known as the Bloch equations.
\subsection{Bloch Equations}
\subsection{Spin Relaxation Times}
\subsection{Spin Packets}

\section{Instrumentation}
\subsection{Pulse Sequences and Measurement Techniques}
\subsubsection{The Refocused Spin Echo}
The Hahn Echo sequence consists of two pulses, the $\pi/2$ pulse and the $\pi$ pulse, separated in time by $\tau$: $\pi/2 - \tau - \pi - \tau - echo$. Initially, the macroscopic magnetization of the spin system is aligned along $\vec{B_0}$: $\vec{M}_0=M_Z~\vec{e_Z}$. The $\pi/2$ microwave pulse has such length $t_{\pi/2}$ and amplitude $B_1$ that, during the pulse, $\vec{M}$ nutates to the $xy$ plane, where it keeps precessing about $\vec{e_Z}$ after the end of the pulse. The difference in local environments for each individual spins in the spin packet, as well as the interactions between the spins, that make up $\vec{M}$, leads to slightly different precession frequencies $\omega_L^i$ of the spins. After some time $\tau$, the difference in the precession frequencies translates into the differences in phases so that the vector sum of the excited spins averages down to $\vec{0}$ for sufficiently long $\tau$. In other words, the excited spin packet dephases with time. The dephasing due to different local spin environments can be reversible if the deviations of the precession frequencies do not depend on time, as is the case for separated electrons in an inhomogeneous solid. In such case, a $\pi$ pulse can be applied to the spin system to flip every single spin in the dephased spin packet by 180$\deg$ in a plane containing $\vec{e}_Z$, so that the spins keep precessing in the $xy$ plane, but the direction of precession is inverted for them, leading to the effect that is opposite to the initial dephasing. So a $\tau$ after the $\pi$ pulse excites the spin packet, the accumulated phase differences become the smallest and the packet recovers its macroscopic magnetization $\vec{M}$ that oscillates in the $xy$ plane with $\langle\omega_L^i\rangle$ and can be detected. The recovered $\vec{M}$ at $t=\tau$ after the $\pi$ pulse is called the refocused spin echo. The difference in $\omega_L^i$ leads to a further dephasing of the considered spin packet and to the vanishing of $\vec{M}$.\\
\subsubsection{Spin Echo Decay and Phase Memory Time}
\subsubsection{Inversion Recovery and Spin-Lattice Relaxation Time}

\subsection{Broad-Band Excitation and Instantaneous Diffusion}
In Section /// it is shown that in a densely packed radical system, as in a TEMPO-Salen cathode film, the phase memory time can be shorter than $T_m\leq100$~ns. That is, the spin echo is decaying by $e\approx3$ at $t=100$~ns. The short phase memory time limits the duration of the pulse sequence at which the echo is detectable. For a $\pi/2-\tau-\pi-\tau-echo$ sequence, with a hardware limitation on $\tau\geq t_d\approx100$~ns, the pulse sequence is longer than $t>200$~ns. By this time, the spin echo decreases by $e^2\approx7$ and may be comparable to noise. The limitations imposed by the finite $T_m$ and $t_d$ force one to use shorter microwave pulses. 
\par
A short microwave pulse may have a spectral width comparable to the width of the observed spectrum. According to the Fourier theorem, the spectral width of a pulse is inversely proportional to the pulse length: $\Delta\omega\sim 1/t_p$. A spectrum of a 100~ns long rectangular pulse shown in Figure /// is ///MHz wide (FWHM). 

\begin{figure}[h]
\center
	\includegraphics[width=1\textwidth]{./pulse/figures/FSE_DTBS_FSE_RELAX_T1Tm.pdf}
	\caption{XXX}
	\label{fig:spectra_of_pulses}
\end{figure}


\section{Pulsed EPR Spectroscopy of a charged pDiTBuS Cathode film}

\subsection{Field Swept Echo of a charged pDiTBuS Cathode film}
\begin{figure}[h]
\center
	\includegraphics[width=1\textwidth]{./pulse/figures/FSE_DTBS_FSE_RELAX_T1Tm.pdf}
	\caption{XXX}
	\label{fig:Figure_FSE1}
\end{figure}


\subsection{Estimation of Local Spin Concentrations with Instantaneous Diffusion}
\subsection{Spin Relaxation in a charged pDiTBuS Cathode Film}

\section{Pad{\'e}-Laplace Deconvolution of Polyexponential Decay Signals}
\label{sec:pade-laplace}
The echo decay and inversion recovery transients measured in the corresponding experiments may contain multiple exponential decay components. The conventional method of determining the distribution of the decay components in a transient decay is the Laplace inversion, where the signal in the time domain $s(t)$ is converted into its Laplace image $L(p)=\int\limits_{0}^{+\infty}s(t)e^{-pt}\mathrm{d}t$ in the time-constant domain $p=1/t$, where the peaks of $L(p)$ give the decay constants that make up the signal. However, for the noisy signal, the direct calculation of the Laplace transform brings in artifacts that drastically vary with the noise. The signal-to-noise ratio (SNR) of the recorded data makes it difficult to apply the Laplace inversion to determine the number of the decay components, as the Laplace transform is unstable at that SNR. \\ 


The Pad{\'e}-Laplace method comes useful for analyzing noisy polyexponential decays as it was demonstrated in Ref.~\cite{Hellen_2005}. The idea of the Pad{\'e}-Laplace method is to analyze the Pad{\'e} approximation of the Taylor expansion of the $L(p)$ in the vicinity of one of the expected decay constants $p_0$, rather than considering the $L(p)$ fully. This way of signal decomposition is stable against the noise for SNR$<$10 (see Figure~\ref{fig:Figure_S7}). The number of exponents detected by the Pad{\'e}-Laplace method as well as their locations in the $p$ space may vary depending on the expansion point $p_0$. We considered $p_0=1/t_{1/2}$ where $t_{1/2}$ is the time at which the signal amplitude halves.\\

We implemented the following algorithm to detect the number of exponents in the decaying transient $s(t)$:\\

First, a point $p_0=1/t_{1/2}$ was chosen, at which $s(t)$ halves.\\

Then, 11 coefficients of the Taylor expansion of the Laplace transform $L(p)$ were calculated in the vicinity of $p \rightarrow p_0$\\

\begin{equation}
L(p)\vert_{p \rightarrow p_0} = \sum_{n=0}^{11} d_i (p-p_0)^i
\end{equation}
with
\begin{equation}
d_i = \frac{1}{i!}\left(\frac{d^{(i)}L}{dp^{(i)}}\right)_{p=p0}
\end{equation}
where the derivatives $\frac{d^{(i)}L}{dp^{(i)}}$ are computed numerically at the point $p=p_{0}$ from the discrete signal $s(t) = (t_j,f_j),~j=1~...~M$:
\begin{equation}
\frac{d^{(i)}L}{dp^{(i)}}= \sum_{j=2}^{M-1}(-t_j)^ie^{(-p_0t_j)}f_j + \frac{1}{2}\left((-t_1)^ie^{(-p_0t_1)}f_1+(-t_M)^ie^{(-p_0t_M)}f_M \right)
\end{equation}


Then the Taylor expansion for $L(p)$ was approximated in the vicinity of $p_0$ with the Pad{\'e} polynomials a(p) and b(p) of orders $n=5$ and $n=6$ respectively:\\

\begin{equation}
\label{eq:pade}
L(p)\vert_{p \rightarrow p_0} = d_0+d_1(p-p_0)+\dots+d_{11}(p-p_0)^{11}=\frac{a_0 + a_1(p-p_0) + \dots + a_5(p-p_0)^5}  {1 + b_1(p-p_0)+ \dots+ b_6(p-p_0)^6}
\end{equation}

The equation \ref{eq:pade} is multiplied by the the denominator and the prefactors at $(p-p_0)^{n}$ for $n=6\dots11$ are compared. This defines a system of linear equations on coefficients $b_i$:\\


\begin{equation}\label{eq:matrix}
  \begin{pmatrix}
    d_{5} & d_{4} & d_{3} & d_2 & d_1 & d_0\\
    d_6 & d_5 & d_4 & d_3 & d_2 & d_1\\
    d_7 & d_6 & d_5 & d_4 & d_3 & d_2\\
    d_8 & d_7 & d_6 & d_5 & d_4 & d_3\\
    d_9 & d_8 & d_7 & d_6 & d_5 & d_4\\
    d_{10} & d_9 & d_8 & d_7 & d_6 & d_5
  \end{pmatrix}
       \begin{pmatrix}
   b_1\\
   b_2\\
   b_3\\
   b_4\\
   b_5\\
   b_6
   \end{pmatrix}
  =
    \begin{pmatrix}
   -d_6\\
   -d_7\\
   -d_8\\
   -d_9\\
   -d_{10}\\
   -d_{11}
   \end{pmatrix}
\end{equation}

From \ref{eq:matrix}, $b_i$ are found. The coefficients $a_i$ are found from ${d}$ and ${b}$.\\

Finally, the Pad{\'e} approximation $P=\frac{a}{b}$ is constructed and its poles are analyzed. The position of the poles reveal the decay constants. The residues at the poles may give the amplitudes of the decay components, but this was not used in the present study.

\begin{figure}[ht!]
\center
	\includegraphics[width=1\textwidth]{./pulse/figures/Figure_S7.pdf}
	\caption{Pad{\'e}-Laplace deconvolution of the mock data containing two decay constants with various signal-to-noise (SNR) ratios.}
	\label{fig:Figure_S7}
\end{figure}


\newpage
\subsection{Pad{\'e}-Laplace Deconvolution of the Echo-Decay and Inversion-Recovery Transients}
\label{pade_laplace_T2}
We used the Pad{\'e}-Laplace method to determine the number of the decay constants in the signals and used multiexponential fits to adjust the decay constants and their amplitudes. For the echo decay transients the best fits were monoexponential fits, even though the Pade-Laplace analysis initially showed biexponential behavior for pDiTBuS with major contribution from the fast decay constants in the order of the detector's dead time. All recorded echo decay transients contain oscillations due to the ESEEM effect. The oscillations were excluded from the analysis as shown in the 'fit area' on the plots in Figure~\ref{fig:Figure_S9}, Figure~\ref{fig:Figure_S11} and Figure~\ref{fig:Figure_S13}.

\newpage
\subsubsection{Echo Decay in 50 mM TEMPOL}
\begin{figure}[ht!]
\center
	\includegraphics[width=1\textwidth]{./pulse/figures/Figure_S8.png}
	\caption{Pad{\'e}-Laplace deconvolution of the field-swept spin echo decay in a frozen 50~\si{\milli\Molar}  solution of TEMPOL in Dichloromethane:Acetonitrile glass (3:1). One decay component detected with Pade-Laplace (triangles, right panel). Monoexponential fit (squares for faster component, circles for slower component, right panel). Temperature 5K.}
	\label{fig:Figure_S8}
\end{figure}

\begin{figure}[ht!]
\center
	\includegraphics[width=0.9\textwidth]{./pulse/figures/Figure_S9.pdf}
	\caption{Fits of the echo decay transient in the frozen 50~\si{\milli\Molar}  TEMPOL solution at the central spectral peak ($m_I=0$, 342~mT). Pad{\'e}-Laplace deconvolution the transient vs. free monoexponential fit vs biexponential fit. Temperature 5K. Data was fit in the 'fit area' region. ESEEM oscillations were excluded from the data for fit and Pade-Laplace analysis.}
	\label{fig:Figure_S9}
\end{figure}


\newpage
\subsubsection{Echo Decay in DiTBuS 95\% ESOC}
\begin{figure}[h]
\center
	\includegraphics[width=1\textwidth]{./pulse/figures/Figure_S10.pdf}
	\caption{Pad{\'e}-Laplace deconvolution of the field-swept spin echo decay in the pDiTBuS film at 95\%~ESOC. Temperature 5K.}
	\label{fig:Figure_S10}
\end{figure}


\begin{figure}[ht!]
\center
	\includegraphics[width=0.9\textwidth]{./pulse/figures/Figure_S11.pdf}
	\caption{Fits of the echo decay transient in the pDiTBuS film at 95\%~SoC at the central spectral peak ($m_I=0$, 342~mT). Pad{\'e}-Laplace deconvolution the transient vs. free monoexponential fit vs biexponential fit. Temperature 5K. Data was fit in the 'fit area' region. ESEEM oscillations were excluded from the data for fit and Pade-Laplace analysis.}
	\label{fig:Figure_S11}
\end{figure}


\newpage
\subsubsection{Echo Decay in DiTBuS 85\% ESOC}

\begin{figure}[h]
\center
	\includegraphics[width=1\textwidth]{./pulse/figures/Figure_S12.png}
	\caption{Pad{\'e}-Laplace deconvolution of the field-swept spin echo decay in a pDiTBuS film at 85\%~ESOC. Temperature 5K.}
	\label{fig:Figure_S12}
\end{figure}

\begin{figure}[ht!]
\center
	\includegraphics[width=0.9\textwidth]{./pulse/figures/Figure_S13.pdf}
	\caption{Pad{\'e}-Laplace deconvolution, biexponential and monoexponential fits of the spin echo decay in a pDiTBuS film at 85\%~SoC at the $m_I=0$ spectral position. Temperature 5K. Data was fit in the 'fit area' region. ESEEM oscillations were excluded from the data for fit and Pade-Laplace analysis.}
	\label{fig:Figure_S13}
\end{figure}



\newpage
\subsubsection{Inversion Recovery in 50 mM TEMPOL}
\label{esi:pade_laplace_T1}
\begin{figure}[h]
\center
	\includegraphics[width=1\textwidth]{./pulse/figures/Figure_S14.png}
	\caption{Pad{\'e}-Laplace deconvolution of the field-swept inversion recovery in a frozen 50~\si{\milli\Molar}  solution of TEMPOL in the Dichloromethane:Acetonitrile glass (3:1). Two decay components detected with Pade-Laplace (separated poles in the Pad{\'e}-Laplace approximation, third panel). Biexponential fit (circles for faster component, squares for slower component, right panel). Temperature 5K.}
	\label{fig:Figure_S14}
\end{figure}


\begin{figure}[ht!]
\center
	\includegraphics[width=0.9\textwidth]{./pulse/figures/Figure_S15.pdf}
	\caption{Fits of the inversion recovery transient in the frozen 50~\si{\milli\Molar}  TEMPOL solution at the central spectral peak ($m_I=0$, 342~mT). Pad{\'e}-Laplace deconvolution the transient and a biexponential fit. Temperature 5K.}
	\label{fig:Figure_S15}
\end{figure}



\newpage
\subsubsection{Inversion Recovery in DiTBuS 95\% ESOC}
\begin{figure}[h]
\center
	\includegraphics[width=1\textwidth]{./pulse/figures/Figure_S16.pdf}
	\caption{Pade-Laplace deconvolution of the field-swept inversion recovery in a pDiTBuS film at 95\%~ESOC. Temperature 5K.}
	\label{fig:Figure_S16}
\end{figure}

\begin{figure}[ht!]
\center
	\includegraphics[width=0.9\textwidth]{./pulse/figures/Figure_S17.pdf}
	\caption{Fits of the inversion recovery transient in the 95\% ESOC pDiTBuS film at the central spectral peak ($m_I=0$, 342~mT). Pad{\'e}-Laplace deconvolution the transient and a biexponential fit. Temperature 5K.}
	\label{fig:Figure_S17}
\end{figure}


\newpage
\subsubsection{Inversion Recovery in DiTBuS 85\% ESOC}
\begin{figure}[h]
\center
	\includegraphics[width=1\textwidth]{./pulse/figures/Figure_S18.pdf}
	\caption{Pade-Laplace deconvolution of the field-swept inversion recovery in a pDiTBuS film at 85\%~ESOC. Temperature 5K.}
	\label{fig:Figure_S18}
\end{figure}


\begin{figure}[ht!]
\center
	\includegraphics[width=0.9\textwidth]{./pulse/figures/Figure_S19.pdf}
	\caption{Pade-Laplace deconvolution and biexponential fit of the inversion recovery in a pDiTBuS film at 85\%~ESOC at the $m_I=0$ spectral position. Temperature 5K. Data is inverted and scaled before fitting.}
	\label{fig:Figure_S19}
\end{figure}




\subsection{Detection of Domains with Poor Conductivity}
\label{domains_distinction_by_relaxation}

\subsection{Towards Imaging of Spin Concentration in Battery Electrodes}
One can obtain a spatially resolved image of the spin concentrations inside a battery electrode by encoding the position with a gradient of the magnetic field. With the procedure described in Section~\ref{domains_distinction_by_relaxation} and using a pair of electromagnetic coils to superimpose a gradient of $B_0$ one can not only measure the spin concentrations that are present in the electrode, but also to locate the electrochemically inactive domains and to visualize the conductive paths throughout the electrode.

\subsection{Unusual Peak Ratios in a Highly Charged Cathode Film}




