\chapter{Conclusions and Outlook}

\paragraph{EES}
%
Our results show that electron spins can be used as unique local probes for elucidating redox reactions associated with charging and discharging of ORBs with active electrode materials prepared by electropolymerization. EPR spectroscopy can thus be employed to analyze these processes, identify performance-limiting loss mechanisms, and eventually help developing strategies for making polymer batteries powerful contenders on the path towards sustainable electrochemical power sources. 

\par
The fact that we can observe a strong influence of the electrical potential on the in-situ EPR spectrum at ambient conditions provides the basis for systematically studying the redox-active sites in different states of the active electrode materials. For instance, changes in the film structure, e.g.\ caused by degradation, can be detected in in-situ cwEPR experiments (cf.\ Fig.~\ref{fig:Figure_5}) as nitroxides in different environments are associated with unique and characteristic EPR signatures (cf.\ Fig.~\ref{fig:Figure_2}). Much more detailed and specific information about degradation processes can, however, be extracted from advanced (pulse) EPR experiments which usually require low temperatures, mostly to increase spin-relaxation times. We believe that the possibility of performing low-temperature pEPR measurements is the key factor that is mandatory for a widespread use of advanced EPR techniques for ORB research.

\par
The electrochemical cells for in-situ studies, as described in this article, were designed in such a way that they fit into a standard EPR tube. They are thus compatible with (low-temperature) experiments using conventional pEPR instrumentation. This opens up the intriguing perspective of exploiting the coupling between neighboring electron spins as well as the hyperfine coupling between electron and nuclear spins in close vicinity to determine the coupling (and thus the distance) between redox-active centers and the degree of delocalization for charge carriers on the conjugated polymer backbone. In particular, it is conceivable to find out whether inactive redox centers, which are not changing their redox state upon charging and discharging, can be found in clusters or are rather isotropically distributed in the cathode films. This provides the possibility to identify inactive redox sites that reduce the maximum attainable cell capacity. The pEPR measurements can either be performed ex situ (without electrolyte, cf.\ Fig.~\ref{fig:Figure_7} and Fig.~\ref{fig:Figure_8}) or on samples that are flash-frozen while the desired electrical potential is applied to the cell.

\par
The experiments presented here are by no means limited to the electrochemical cells used in this study. Specifically, they could also be used for investigating all-organic solid-state batteries such as the recently reported and very promising cells comprising polymeric anode and cathode materials as well as gel electrolytes based on ionic liquids.\cite{muench2021_jps} In particular, low-temperature spin-echo experiments also hold promise for elucidating electron transfer processes in redox flow batteries.\cite{zhao2021_jacs}

\par
A strategy for overcoming sensitivity limitations of advanced EPR experiments is to combine them with electrical readout schemes and make Electrically Detected Magnetic Resonance (EDMR) applicable to study redox and transport processes in polymer batteries. EDMR possesses a sensitivity many orders of magnitude higher than conventional EPR measurements.\cite{mccamey2006_apl} The possibility of using EDMR techniques for battery characterization at ambient conditions would also open up entirely new opportunities for imaging techniques. EDMR imaging could provide a detailed picture of the spatial distribution of active and inactive redox centres and, in general, usefully complement information obtained through imaging techniques based on conventional EPR~\cite{sathiya2015_natcomm, dutoit2021_natcomm, geng2021_chemofmat} or nuclear magnetic resonance.\cite{ilott2016_pnas}\par



\paragraph{JMRO}
\rs{We connected the Coulomb-counting state of charge of \q{a pDiTBuS} film to the number of reduced and thus paramagnetic TEMPO moieties \w{by} measuring the number of spins at various SoC with cwEPR. The spin count\q{ing} performed with cwEPR allowed us to determine the average spin concentration in a film at a given ESOC. The average spin concentration in the film changes upon oxidation between \q{$(5\pm3)\times10^{20}$} and \q{$(3\pm2)\times10^{19}$~cm$^{-3}$}. At high SoC, the number of injected charges fits well to the number of EPR-detected charges. However, the lower the SoC, the larger fraction of the injected charges becomes EPR silent. At SoC~0\%, the fraction of EPR silent charges is 78~\%. The concentration of injected charges at SoC~0\% corresponds to the average distance between the charges of $d=0.8$~nm, which may allow for a formation of singlet spin states (with $S=0$)~\cite{Behrends2010}. While all ESOC yield a measurable cwEPR signal, at ESOC $\leq 49$~\% no spin echo can be observed.\\} 




%\ik{Due to a dense packing of charges in discharged pDiTBuS (at low ESOC) with the average inter-spin distance of $d=0.8$~nm, a noticeable magnetic dipolar coupling is present between their unpaired spins, that promotes a formation of EPR-silent pairs of charges with $S=0$. The bosonic nature of the coupled charges suggests an efficient packing and transport of charge due to the possible condensation effect. At ESOC~0\% upto 78\% of the injected charges were found in the $S=0$ state, that means, the formation of $S=0$ states is possible both at the pNiSalen backbone and at some neighboring TEMPO$^{\bullet}$.}



\ik{The strong inter-spin interactions in the discharged battery cathode drastically reduce the phase memory time $T_m$ of the spin packets probed by \q{p}EPR, especially for the domains with the highest concentration of spins. For ESOC~$<(49\pm3)~\%$ with the average spin concentration \w{$\langle n \rangle > (3\pm2)\times10^{20}$~cm$^{-3}$}, the $T_m$ becomes shorter than the spectrometer dead time $t_d=120$~ns, thus by the time of detection, the spin echo becomes indistinguishable from \q{the} noise. The low values of $T_m$ represent a challenge to measure \q{p}EPR in materials with $\langle n \rangle > 10^{20}$~cm$^{-3}$. The alternative, dead-time-free detection schemes\cite{Schweiger1988,Granwehr2001,Nasibulov2017,Kraffert2017} may be employed to overcome the limitation imposed by short $T_m$ in energy storage materials.} \rs{With the dead-time-free detection, the echo decay transients can be measured at earlier times $\tau<t_d$, allowing for detecting the spin echoes from the domains with higher spin concentrations. That would extend the range of the observable local spin concentration $C$ and allow for precise \q{identification} of the charged domains in a battery electrode.}

\ik{Spin-lattice relaxation times $T_1$, measured in the inversion recovery experiments, have revealed two types of domains in a partially charged pDiTBuS film. \q{Both domain types may have comparable spin concentrations and therefore similar $T_m$.} The \q{``conductive''} domains with shorter $T_1$ are likely to have a more efficient charge transfer kinetics between the TEMPO and the conductive pNiSalen backbone of pDiTBuS, while the \q{``non-conductive''} domains with longer $T_1$ have a weaker interaction between TEMPO and the environment\q{,} which implies a lower probability for electron transfer between the TEMPO and the backbone. \rs{However,} the separation between the two $T_1$ distributions is rather small and the existence of the two domain types is debatable. The information extracted from the inversion recovery measurements can be used to differentiate between the domains of various charge transport efficiency, which can be used for \rs{optimizing the electrode material, such that one predominant type of domains is growing during electropolymerization}.\\}

The ability to detect a spin echo signal in a material with densely packed spins, as pDiTBuS, opens the intriguing perspective to apply advanced pulsed \q{p}EPR techniques to study novel battery materials on a molecular level. For instance, charge delocalization and intermolecular interactions in organic semiconductors can be observed with \q{p}EPR methods, \q{particularly} including double resonance techniques\cite{Tait2021}. ENDOR was used for measuring charge delocalization in molecular \ik{semiconductor} dopants~\cite{arvind2020_jpcb}; electron-electron double resonance (ELDOR) was used to measure interspin distances between spin-labeled sites in a protein~\cite{jeschke2012_annrevphyschem} and to observe transient radical formation in photosystems~\cite{Bittl2005}. The presented analysis of the instantaneous diffusion caused by the short microwave pulses and substantial inter-spin interactions \ik{provides information on} the local spin concentrations in the cathode at various states of charge. Combined, the cwEPR, electrochemical and \rs{pEPR} data help one to picture the charging of the battery electrode on a microscopic level, allowing for a detailed monitoring of the processes that lead to degradation of the electrode capacity, such as the formation of electrically disconnected domains \q{(``dead mass'')}, and to improve the film deposition procedure by monitoring its morphology through the $T_1$ values.



%\ik{We connected the Coulomb-counting state of charge of a pDiTBuS film (SoC) to the number of unpaired electron spins in it (ESOC) using cwEPR spectroscopy. The spin count performed with quantitative cwEPR allowed us to determine the average concentration of unpaired spins in a film at a given SoC. The average concentration of unpaired spins in the film significantly changes upon oxidation between $(5.3\pm1.7)\times10^{20}$ and $(3.0\pm1.5)\times10^{19}$~cm$^{-3}$. ESOC appears to be a universal, material-specific value to characterize the state of charge of a material, unlike the electrochemical methods which yield different SoC values depending on the measurement conditions~\cite{Rezvanizaniani2014}. However, ESOC is insensitive to charges that are coupled to form diamagnetic states with $S=0$. Upon measuring the ESOC values on a discharging pDiTBuS film, we noticed, that the number of unpaired electron spins for every SoC is significantly lower than the number of electrons injected into the film according to Coulomb counting, especially for the deeply discharged states (Table~\ref{tab:Table1}, “charges injected” vs. “spins detected”).


%The discrepancy between the Coulomb counting and ESOC hints that upto 20\% of injected electrons are in the diamagnetic ($S=0$) state at 0\% SoC, while in the fully charged state most of the charges are paramagnetic ($S=1/2$). It was previously found by using UV-Vis spectroscopy and electrochemistry~\cite{Vereshchagin2020}, that a significant fraction of charges injected into a TEMPO-Salen film localizes on the NiSalen backbone. Since we have not detected any EPR signal from NiSalen for the pDitBuS film at any SoC, we assume that the charges on the NiSalen backbone form positively charged bipolarons in singlet states with total spin $S=0$. The 20~\% fraction of the EPR-silent species at 0\% SoC goes well with the predicted contribution of NiSalen into the total electrochemical capacity of pDiTBuS~\cite{Vereshchagin2020}. The formation of positively charged singlet bipolarons was previously observed in bare pNiSalen with operando cwEPR upon oxidation~\cite{Dmitrieva2018}.}


%While all ESOC yield a measurable cwEPR signal, at ESOC $\leq$ 49\% no spin echo can be observed. ESOC for a given material is, therefore, a useful measure of the applicability of the conventional pulse EPR toolbox to study that material. The high spin concentration at low ESOC leads to strong inter-spin couplings that result in a phase memory time shorter than the spectrometer dead time($T_m<120$~ns). For the oxidized states with spin concentrations below $2\times10^{20}$~cm$^{-3}$ the EDFS spectrum of pDiTBuS is detectable and clearly deviates from the EDFS spectrum of noninteracting TEMPO fragments. The short $T_m$ in charged pDiTBuS limits the length of the used microwave pulses that leads to a broad-band excitation of the spectrum and introduces instantaneous diffusion. The effect of instantaneous diffusion is well pronounced for the 85\% ESOC. The simulation of the effect of instantaneous diffusion allowed us to estimate the lower value of the local spin concentration in the oxidized pDiTBuS film without any measurements of the volume of the film. %For a TEMPOL sample with a known spin concentration of $6.3\times10^{19}$~cm$^{-3}$ the simulation of instantaneous diffusion resulted in a spin density of $2.6\times10^{19}$~cm$^{-3}$ which is 13\% lower. 


%\sout{For pDiTBuS at \ik{highly oxidized sates (ESOC~85\% and 95\%)}, the instantaneous diffusion corresponds to \ik{local spin concentrations $C\geq(0.1\ldots0.3)\times10^{20}$~cm$^{-3}$,} while the cwEPR spin count yields higher values of $\langle n \rangle = (0.3\ldots0.8)\times10^{20}$~cm$^{-3}$ \ik{respectively. The Coulomb counting predicts a concentration of the injected charges that is similar to the cwEPR data: $n_{CC}\leq(0.4\ldots0.9)\times10^{20}$~cm$^{-3}$. Thus, all charges injected into the film at high ESOC are in the paramagnetic, $S=1/2$ state. Given that the corresponding cwEPR spectrum contains only the Nitroxide component, we conclude that all injected charges localize on the TEMPO groups in this ESOC}.} \ik{The values of local spin concentration $C$ determined from the simulations of the instantaneous diffusion are lower than the average values $\langle n \rangle$, as} spins at higher concentrations have faster relaxation times and are not detected by pulse EPR, because of the finite dead time of the spectrometer (see Figure~\ref{fig:Figure_S28}). Therefore, local spin concentrations determined with pulsed EPR are underestimated.


%The EDFS spectrum for the 95\%~ESOC is distorted, but cannot be simulated by assuming only the effect of the instantaneous diffusion. The instantaneous diffusion is not solely responsible for the distortion of the EDFS spectrum at 95\% ESOC. The conductive nature of pDiTBuS, high electron spin density and early saturation of cwEPR may cause an effect similar to the Overhauser effect in metals~\cite{Overhauser1953} that causes dynamic nuclear polarization which affects the populations of the nuclear spin sublevels~\cite{Carver1953,Carver1956,Weber2017,Atsarkin1978}. The effect of dynamic nuclear polarization in pDiTBuS can be observed with quantitative nuclear magnetic resonance, or with electron-nuclear double resonance (ENDOR).\\


%Another potential explanation for the distortion in the ESOC~95\% spectrum may be an additional signal coming from the positive polaron trapped in the conductive pNiSalen backbone. However, the position of the cwEPR signal measured in p-NiSalen does not match the position of the distorted $m_I=+1$ peak measured in pDiTBuS (see simulations in the ESI, Section~\ref{esi:cw_sims}) \ik{and the Coulomb counting data combined with the spin counting indicates that the charges injected into the backbone are likely to couple into EPR silent $S=0$ bipolarons}. Therefore, the distorted shape of the EDFS spectrum cannot be directly attributed to the polaron in pNiSalen.\\


%The measured values of spin memory times $T_m$ lay close to the detection limit of the spectrometer ($T_m\geq100$~ns for pDiTBuS at 85\%~ESOC, Figure~\ref{fig:Figure_5}c, right). The spectrometer dead time limits the minimal pulse separation $\tau$ in a 2 pulse spin echo sequence. The minimal pulse separation sets a higher limit on the spin densities that can be observed with pEPR ($n<10^{21}$~cm$^{-3}$ for $\tau\geq120$~ns). Therefore, only high states of charge (with decreased spin density) yield a detectable pulse EPR signal. This limitation on pulse separation time can be lifted by minimizing the ring-down of the microwave cavity or, possibly, by using the longitudinal detection scheme~\cite{Schweiger1988,Granwehr2001}.\\


%The splitting of the spin-lattice relaxation time constants $T_1$ for charged pDiTBuS into two subsets, detected by the Pad{\'e}-Laplace analysis, indicate two subsystems of TEMPO$^{\bullet}$ in the film, that have distinct local concentrations, for both 85\% and 95\% ESOC.\\ 


%The ability to detect a spin echo signal in a material with densely packed spins, as pDiTBuS, opens the intriguing perspective to apply advanced pulsed EPR techniques to study novel battery materials on a molecular level. For instance, charge delocalization and intermolecular interactions in organic semiconductors can be observed with pulse EPR methods, including the double resonance techniques\cite{Tait2021}. ENDOR was used for measuring charge delocalization in molecular \ik{semiconductor} dopants~\cite{arvind2020_jpcb}; electron-electron double resonance (ELDOR) was used to measure interspin distances between spin-labeled sites in a protein~\cite{jeschke2012_annrevphyschem} and to observe transient radical formation in photosystems~\cite{Bittl2005}. The presented analysis of the instantaneous diffusion caused by the strong interspin interactions allows one to estimate local spin concentrations in the energy storage material at various states of charge and to screen out the formation of electrically disconnected domains upon repeated charge-discharge cycling. The EPR-detected state of charge allows for precise measurement of charge density in the battery electrode without the need of electrochemical characterization, which can be useful to monitor battery health and to understand the formation of the electrochemically inactive 'dead mass'. The discrepancy between SoC and ESOC values may provide information on other performance limiting processes in the electrode. With the limitations considered in this manuscript, the advanced pulse EPR techniques may be applied to measure local spin concentrations inside an ORB electrode, to track the charge transfer pathway and to identify possible loss mechanisms in an electrochemical cell. The in situ pulse EPR measurements may be performed on a fully assembled, EPR compatible electrochemical cell, containing also the anode, the reference electrode and the electrolyte. The SoC of the cell can be changed at room temperature and the pulse EPR measurements can be performed at cryogenic temperatures after the cell is flash-frozen in liquid nitrogen. Such experiment is, however, strongly influenced by the side reactions that take place at the solid-electrolyte interface upon repeated charging and temperature cycling of the cell.






\paragraph{ME}

It is possible to perform a cwEPR experiment on a working organic radical battery.\\
Degradation of electric capacity of a polymer electrode was attributed to a formation of electrochemically disconnected domains in the electrode upon repeated recharging.\\

The significant difference in the number of elementary charges and the number of electron spins in a discharged electrode implies that the closely packed radicals in the discharged electrode pair up into S=0 states. Therefore, a formation of two-particle charge storing unit was observed. The transport of such two-particle units through the electrode film may be more efficient than the single-particle transport, as the total S=0 spin state allows them to recombine with the bipolarons in the NiSalen backbone without obeying the Pauli exclusion principle. The formation of paired spin states between the charge-bearing radicals is increasing the speed of the charge transfer and explains the high charging and discharging rates of TEMPO-Salen RCPs.\\

cwEPR monitoring of the degradation in pDiTS has shown the extensive release of the mono-nitroxide fragments which indicated that the decomposition of the pDiTS electrode upon is due to the breaking of the Salen-TEMPO linkers. The improved molecular structure of the Salen-TEMPO linkers in pDiTBuS by adding the oxygen atoms to the chains has improved the electrochemical stability of the material and also lead to a more efficient electropolymerization, allowing to grow thicker films that have larger capacity and more intense cwEPR signals.\\

Study of self-discharged have demonstrated the possibility to measure the concentration of the released mono-nitroxide shuttles and allowed to connect the concentration of the shuttles to the self-discharge rate.\\

Short spin relaxation times are limiting the application of pulse EPR techniques. Discharged cathode films have high spin concentrations that lead to spin dephasing on a timescale shorter than the dead time of the microwave detector. This limitation can be overcome using longitudinal detection of magnetization.\\

Spectral distortions caused by the instantaneous diffusion due to the broad-band excitation were simulated numerically and used to estimate the local spin concentrations in the cathode. The incomplete spectral information due to the finite dead time of the detector allows to only give the lower estimate of the local spin concentration.\\

The echo decay and inversion recovery transients were shown to be polyexponential signals, with individual components corresponding to certain spin concentrations within the electrode. It was possible to do an analytic deconvolution of the devay signals into its exponential components without specifying the number of components, by using the Pade-Laplace analysis.\\

The observation of distribution of the inversion recovery times with longitudinal detection may further increase the sensitivity of the method for detecting the electrically inactive domains in the film. Combining the polyexponential analysis of the inversion recovery transients with the spatially resolved spectroscopic measurements may enable the electron resonance imaging of spin concentrations in a battery cathode. However, the existence of the S=0 states in the discharged cathodes makes this method applicable only to higher states of charge, where the charges are mostly in the S=1/2, EPR active state.


EPR-compatible, polymer electrochemical cells with solid electrolyte were fabricated. These devices are shown to be more stable and longer charging with smaller current can be used to detect spin-dependent charge transport processes in them, by using the electrically detected magnetic resonance techniques.\\






