\chapter{Conclusions and Outlook}

It is possible to perform a cwEPR experiment on a working organic radical battery.\\
Degradation of electric capacity of a polymer electrode was attributed to a formation of electrochemically disconnected domains in the electrode upon repeated recharging.\\

The significant difference in the number of elementary charges and the number of electron spins in a discharged electrode implies that the closely packed radicals in the discharged electrode pair up into S=0 states. Therefore, a formation of two-particle charge storing unit was observed. The transport of such two-particle units through the electrode film may be more efficient than the single-particle transport, as the total S=0 spin state allows them to recombine with the bipolarons in the NiSalen backbone without obeying the Pauli exclusion principle. The formation of paired spin states between the charge-bearing radicals is increasing the speed of the charge transfer and explains the high charging and discharging rates of TEMPO-Salen RCPs.\\

cwEPR monitoring of the degradation in pDiTS has shown the extensive release of the mono-nitroxide fragments which indicated that the decomposition of the pDiTS electrode upon is due to the breaking of the Salen-TEMPO linkers. The improved molecular structure of the Salen-TEMPO linkers in pDiTBuS by adding the oxygen atoms to the chains has improved the electrochemical stability of the material and also lead to a more efficient electropolymerization, allowing to grow thicker films that have larger capacity and more intense cwEPR signals.\\

Study of self-discharged have demonstrated the possibility to measure the concentration of the released mono-nitroxide shuttles and allowed to connect the concentration of the shuttles to the self-discharge rate.\\

Short spin relaxation times are limiting the application of pulse EPR techniques. Discharged cathode films have high spin concentrations that lead to spin dephasing on a timescale shorter than the dead time of the microwave detector. This limitation can be overcome using longitudinal detection of magnetization.\\

Spectral distortions caused by the instantaneous diffusion due to the broad-band excitation were simulated numerically and used to estimate the local spin concentrations in the cathode. The incomplete spectral information due to the finite dead time of the detector allows to only give the lower estimate of the local spin concentration.\\

The echo decay and inversion recovery transients were shown to be polyexponential signals, with individual components corresponding to certain spin concentrations within the electrode. It was possible to do an analytic deconvolution of the devay signals into its exponential components without specifying the number of components, by using the Pade-Laplace analysis.\\

The observation of distribution of the inversion recovery times with longitudinal detection may further increase the sensitivity of the method for detecting the electrically inactive domains in the film. Combining the polyexponential analysis of the inversion recovery transients with the spatially resolved spectroscopic measurements may enable the electron resonance imaging of spin concentrations in a battery cathode. However, the existence of the S=0 states in the discharged cathodes makes this method applicable only to higher states of charge, where the charges are mostly in the S=1/2, EPR active state.


EPR-compatible, polymer electrochemical cells with solid electrolyte were fabricated. These devices are shown to be more stable and longer charging with smaller current can be used to detect spin-dependent charge transport processes in them, by using the electrically detected magnetic resonance techniques.\\






