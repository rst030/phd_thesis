\chapter{Operando EPR Spectroscopy of Energy Storage Materials}
\section{Electron Paramagnetic Resonance}
\subsection{Spin Hamiltonian}
\paragraph*{}
The electron bears an internal angular momentum that is called spin. Spin combines with the charge of the electron to endow the electron with a magnetic moment. The magnetic moment of the electron is quantized: $\mu_e=\mu_BgS$ \cite{SternGerlach1922}, where $S$ is the spin quantum number, the eigenvalue of the spin operator $\vec{\hat{S}}$, that equals $S=1/2$ for an electron. When an electron is placed in a static magnetic field $\vec{B_0}=B_0 \vec{e_z}$, its magnetic moment precesses about the field direction with the Larmor frequency $\omega_L = \gamma B_0$, where $\gamma=\frac{g_e\mu_B}{\hbar}=28.025$~GHz/T is the gyromagnetic ratio of the electron and $g_e$ is the electron g factor. The projection of the electron's magnetic moment on the direction of the magnetic field can take only discrete values between $-S=-1/2$ and $S=1/2$, so that the eigenvalues of the $z$ component of the spin operator are also discrete: $\hat{S}_Z\vert{\uparrow\rangle}=+\frac{\hbar}{2}\vert{\uparrow\rangle}$, $\hat{S}_Z\vert{\downarrow\rangle}=-\frac{\hbar}{2}\vert{\downarrow\rangle}$. The two eigenfunctions of $\hat{S}_Z$ are called the spin-up state $\vert{\uparrow\rangle}$ and the spin-down state $\vert{\downarrow\rangle}$. The two corresponding eigenvalues $\pm\frac{\hbar}{2}$ define the energy difference between the states $\vert{\uparrow\rangle}$ and $\vert{\downarrow\rangle}$, that is known as the Zeeman splitting.

\paragraph*{}
The energy of an unpaired electron placed in the external magnetic field $\vec{B_0}$ is the eigenvalue of the spin Zeeman Hamiltonian: $\hat{H}_{EZ} = \mu_B g\vec{B}_0\cdot\vec{\hat{S}}$. In the laboratory frame of reference $\vec{B_0}\parallel\vec{e_z}$, $\left[\hat{H}_{EZ},\hat{S}_z\right]=0$, so $\hat{H}$ and $\hat{S}_Z$ share the eigenfunctions $\vert{\uparrow\rangle}$ and $\vert{\downarrow\rangle}$. The Zeeman energies of the electron are $E_{EZ}^{\pm} = \pm \frac{\hbar}{2}\mu_B g B_0$. 

\paragraph*{}
A proton also bears an internal angular momentum $S=1/2$ that results in a magnetic moment $\mu_p = \mu_e\frac{m_e}{m_p}$, that is $\frac{m_e}{m_p}\approx1836$ times smaller than the electron's magnetic moment. A neutron bears no charge but still has an internal angular momentum $S=1/2$. An atomic nucleus that consists of protons and neutrons can have a magnetic moment, depending on the mutual orientations of their spins and on the nuclear charge. A nitrogen nucleus has 7 protons and 7 neutrons that total in a nuclear spin $I=1$ which, with the g factor for the nitrogen nucleus $g_N$, results in the nuclear magnetic moment of $\mu_N=\mu_Bg_N\frac{m_e}{m_N}I$ that splits into three Zeeman energy levels corresponding to $I=-1,0,+1$, analogously to the electron with $S=1/2$. The nuclear Zeeman splitting is more than three orders of magnitude weaker than the electron Zeeman splitting because of the difference in the masses of the particles.

\paragraph*{}
The magnetic moments of an electron and a magnetic nucleus, such as nitrogen, couple in the hyperfine interaction~\cite{Schweiger2001_hfi}: $H_{HF}=\vec{\hat{S}}\textbf{A}\vec{\hat{I}}=H_F+H_{DD}$ with the hyperfine tensor $\textbf{A}$. The isotropic part $H_F=a_{iso}\vec{\hat{S}}\vec{\hat{I}}$, or the Fermi contact interaction, scales with the probability density of the electron at the position of the nucleus $a_{iso}=\frac{2}{3}\frac{\mu_0}{\hbar}g_e\beta_eg_n\beta_n\vert\psi(0)\vert^2$. The anisotropic part $H_{DD}=\vec{\hat{S}}\textbf{T}\vec{\hat{I}}$ with the dipolar coupling tensor $\textbf{T}$ takes into account the anisotropic dipole-dipole coupling between the magnetic moments of the electron and the nucleus.

\paragraph*{}
The nitrogen nucleus has a spin greater than 1/2 which alters the charge distribution within the nucleus which gives rise to a non-vanishing nuclear electrical quadrupole moment $Q$. The interaction between the asymmetrically distributed charge and the gradient of the electric field at the nucleus is given by the nuclear quadrupole Hamiltonian $H_{NQ}=\vec{\hat{I}}\textbf{P}\vec{\hat{I}}$ with the nuclear quadrupole tensor $\textbf{P}$ that describes the coupling of the nuclear quadrupole moment to the electric field gradient.

\paragraph*{}
In a system of closely placed electrons, such as in a film of densely packed nitroxide radicals, the electron orbitals may overlap significantly and the radicals may exchange electrons. The energy required to exchange the electrons is called the Heisenberg exchange coupling $H_{exch} = \vec{\hat{S_1}}\textbf{J}\vec{\hat{S_2}}$, that becomes considerably large at inter-spin distances below $r<1.5$~nm or with a large spin delocalization~\cite{Schweiger2001_exch}. The positive $\textbf{J}$ corresponds to a weak coupling between $S_1$ and $S_2$ which leads to an antiferromagnetic or antiparallel alignment of spins with a total $S=0$, whereas the negative $\textbf{J}$ causes the strong inter-spin coupling which leads to a ferromagnetic alignment with $S=1$.

\paragraph*{}
The dipole-dipole interaction between the two neighboring electron spins contributes one more term to the spin Hamiltonian: $H_{dd} = \vec{\hat{S_1}}\textbf{D}\vec{\hat{S_2}}$ that depends on the distance between the spins. 


 